\documentclass[letterpaper, 11pt]{article}

% set version variable
\newcommand{\versionnumber}{0.1}

% russian language
\usepackage[utf8]{inputenc}
\usepackage[T2A]{fontenc}
\usepackage[english, russian]{babel}

% math
\usepackage{mathtools}
\usepackage{amsmath}

\usepackage{amssymb} % some math symbols
% abs function
\DeclarePairedDelimiter{\abs}{\lvert}{\rvert}

% enumerate
\usepackage{enumerate}

% set type and margins of the page
\usepackage{geometry}  % document margins
\geometry{letterpaper, left=1.4in, right = 1.4in, top = 1.7in, bottom = 1.7in}

% color links in content
\usepackage{hyperref}
\hypersetup{
    colorlinks=true,
    linkcolor=red,
    urlcolor=blue,
    linktoc=all
}

% indent at first \par after section
\usepackage{indentfirst}

% fixed table and figures in section
\usepackage{float}

% colors
\usepackage{color}
\usepackage[usenames,dvipsnames]{xcolor}

% paragraph indent
\setlength{\parskip}{0.5em}

\title{\large{Краткий конспект}\\
\LARGE{Лекция 4. Суффиксные деревья}\\
\normalsize версия \versionnumber (\textcolor{NavyBlue}{draft})}
\date{7 марта, 2016}
\author{\underline{Д. Ищенко\thanks{МФТИ}} \and Б. Коварский\footnotemark[1]
\and И. Алтухов\footnotemark[1] \and Д. Алексеев\footnotemark[1]}

\begin{document}
\maketitle
\thispagestyle{empty}
\clearpage

% let's go
\section{Задача поиска k мотивов в геноме}
\par
Отлично справляемся, когда ищем один мотив длиной $p$ в геноме длиной $n$ за $O(p + m)$, но если мотивов $k$ штук, то сложность всех поисков $O((p + n) \cdot k) = O(kn)$. Это уже хуже. Например, при картировании ридов значения $k$ могуть быть порядка $10^6 - 10^8$.
\section{Cуффиксное дерево}
\par
Суффиксы, что из себя представляет дерево. Сколько листьев, сколько узлов, ребер и т.д.
\section{Алгоритм посроения $O(n^2)$}
\par
Как хранить дерево? Как построить дерево за $O(n^2)$. Псевдокод.
\section{Поиск в глубину}
\par
Решаем задачу поиска мотива, как посчитать количество вхождений?
\section{Какие еще задачи можно решить с помощью дерева}
\par
Поиск повтора, поиск максимальной общей подстроки, нечеткий поиск.

\section{Ссылки}

\begingroup
\renewcommand{\section}[2]{}%
\begin{thebibliography}{7}
\bibitem{Frank}
Gusfield D. Algorithms on strings, trees and sequences: computer science and computational biology. – Cambridge university press, 1997.


\end{thebibliography}
\endgroup

\end{document}
